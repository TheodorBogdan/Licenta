\documentclass[a4paper,12pt]{report}
    \title{Service Registry}
\author{Theodor Bogdan Vr\^ancean}
\date {Iunie 2018}

\usepackage[romanian]{babel}

\usepackage[
    backend=biber,
    style=verbose,
    sorting=ynt
]{biblatex}
\addbibresource{References.bib}

\begin{document}
\maketitle 
\chapter{Introducere} 
\section{Arhitectura de microservicii}

Arhitectura de microservicii este o abrodare relativ nou\u a \^ in dezvoltarea de software.
Microserviciile reprezint\u a aplica\c tii mici \c si autonome care lucreaz\u a \^ impreun\u a 
\footcite{buildingMicroservices}. Ele sunt considerate mici relativ la un sistem monolitic care ar
oferi toate func\c tionalit\u a\c tiile de care aplica\c tia are nevoie.Cu toate acestea un mictoserviciu poate oferi orice
fel de func\c tionalit\u a\c ti,incep\^and cu ceva simplu precum desc\u arcarea de fi\c siere, p\^an\u a la 
 complexe precum analizarea imaginilor.
 Aceast\u a aboradare arhitecturala a venit ca o alternativ\u a la arhitectura monolitic\u a, \^in care exist\u a 
 un singur server care satisface toate necesit\u a\c tile unei aplica\c tii.Limit\u arile acestei abord\u ari ies la iveal\u a
 odat\u a cu cre\c sterea aplica\c tiei:
 \begin{itemize}
     \item Din cauza dimensiunii \c si complexit\u a\c tii unui proiect monolitic,
    acesta este dificil de in\c teles \c ,motiv pentru care schimb\u arile sunt mai dificil de facut \c si exist\u a 
    un risc mai mare ca acestea s\u a produc\u a efecte nedorite.Aceste probleme pot fi atenuate printr-un 
    cod de calitate dar acest lucru se \^ inatmpla de prea pu\c tine ori. Schimb\u arile aduse unui microserviciu nu afecteaz\u a alte
    module \c si datorit\u a dimensiunii reduse a acestora, ele sunt \c si mai u\c sor de \^inteles pentru programtori, astfel 
    scade probabilitatea erorilor.Se poate spune ca microserviciile duc un pas mai departe principiul singurei responsabilit\u a\c ti,
    definit de Robert C. Martin.
    \item Pentru dezvoltarea unei aplica\c tii monolitice trebuie sa alegem un tehnologii standardizate care s\u a poata 
    realiza toate cerin\c tele aplica\c tiei. Pe de alta parte,daca avem mai multe microservicii care colaboreaz\u a nu exist\u a aceast\u a limitare,ceea ne permite s\u a alegem 
    unealta cea mai portivit\u a pentru fiecare serviciu.


    

\end{itemize}



 Ele ofer\u a servicii care pot fi utilizate in de c\u atre alte aplica\c tii.
\iffalse Utilizarea unui microserviciu se
 poate asem\u ana cu o clas\u a, apelarea metodelor fiind similara cu trimiterea unei
 cereri c\u atre server.Probabil cea mai mare deosebire intre acestea const\u a in faptul
 c\u a Microserviciile sunt  
\fi
Printre avantajele aduse de microservicii, este faptul c\u a acestea pot fi utilizate de
mai multe aplica\c tii.\^In func\c tie de aceast\u a tr\u as\u atura microserviciile se pot impar\c ti \^in 
trei categorii:
\begin{itemize}
    \item Publice.
    Acestea pot fi puse la dispozi\c tia utilizatorilor pe baza unor chei,gratuite sau nu,
    sau nerestrictionate.
    \item \^Imp\u ar\c tite intre un grup de aplica\c tii.
    Spre exemplu toate aplica\c tiile unei compani pot integra acela\c si serviciu de securitate
    \item Dedicate.
    Acestea sunt folosite doar pentru o singur\u a aplica\c tie 
\end{itemize} 




\chapter{Tehnologii}

    C\# este un limbaj de programare. \^ s \c t \^a

    \section{.Net Framework}
    \section{Entity Framework}
        EntityFramework este un ORM(Object Relational Mapper) open-source pentru platforma .Net.
    \section{Asp.Net}
    \section{MVC pattern}
    \section{Microsoft Sql Server}
    
\chapter{Utilizare}

\end{document}
